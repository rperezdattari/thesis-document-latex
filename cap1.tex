\chapter{Theoretical Framework}
\section{Machine Learning}

Machine Learning (ML) is a discipline that studies ways of finding patterns in data by learning from experience. What this mean, is that an algorithm in charge of accomplishing an specific task is able to progressively improve its performance (learn) by comparing its output with data (experience) in a way that indicates how good/bad is doing. If the algorithm is doing good, then the learning process is finished. If is not doing so good, then the algorithm updates its parameters to match (if possible) its output with what is expected.   
\subsection{Deep Learning}

\subsubsection{Autoencoder}

\section{Policy Learning}
\note{POLICY DESCRIPTION}
\subsection{Reinforcement Learning}
\subsubsection{Deep Reinforcement Learning}
\subsection{Learning from Demonstration}
\subsubsection{The Correspondence Problem}
\subsection{Interactive Learning}
\subsubsection{Evaluative Feedback}
\subsubsection{Corrective Feedback}
\section{Function Approximation}
\subsection{Linear Model of Basis Functions}
\subsection{Artificial Neural Networks}
\section{COrrective Advice Communicated by Humans (COACH)}