\documentclass[upright, contnum]{umemoria}
\depto{Departamento de Ingeniería Eléctrica}
\author{Rodrigo Javier Pérez Dattari}
\title{Interactive Learning with Corrective Feedback for Continuous-Action Policies based on Deep Neural Networks}
\auspicio{FONDECYT Projecto 1161500.}
\date{2018}
\guia{Javier Ruiz del Solar}
\carrera{Magíster en Ciencias de la Ingeniería, Mención Eléctrica}
\carreraeng{Master of Engineering Sciences in Electrical Engineering}
\memoria{Tesis para optar al Grado de \break  Magíster en Ciencias de la Ingeniería, Mención Eléctrica\break \break
Memoria para optar al Título de Ingeniero Civil Eléctrico}
\comision{Carlos Celemin}

\usepackage{lipsum}

\usepackage[latin1]{inputenc}
\usepackage[T1]{fontenc}

\begin{document}

\frontmatter
\maketitle

\begin{abstract}

\end{abstract}

\begin{abstract_eng}
Deep Reinforcement Learning (DRL) has become a powerful strategy to solve complex decision making problems based on Deep Neural Networks (DNNs). However, it is highly data demanding, so unfeasible in physical systems for most applications. In this work, we approach an alternative Interactive Machine Learning (IML) strategy for training DNN policies based on human corrective feedback, with a method called Deep COACH (D-COACH). This approach not only takes advantage of the knowledge and insights of human teachers as well as the power of DNNs, but also has no need of a reward function (which sometimes implies the need of external perception for computing rewards). We combine Deep Learning with the COrrective Advice Communicated by Humans (COACH) framework, in which non-expert humans shape policies by correcting the agent's actions during execution. The D-COACH framework has the potential to solve complex problems without much data or time required. Experimental results validated the efficiency of the framework in three different problems (two simulated, one with a real robot), with state spaces of low and high dimensions, showing the capacity to successfully learn policies for continuous action spaces like in the Car Racing and Cart-Pole problems faster than with DRL.

Deep Reinforcement Learning (DRL) has become a powerful methodology to solve complex decision making problems. However, DRL has several limitations when used in real-world problems (e.g., robotics applications). For instance, long training times are required and cannot be accelerated in contrast to simulated environments, and reward functions may be hard to specify/model and/or to compute. Moreover, the transfer of policies learned in a simulator to the real-world has limitations (reality gap). On the other hand, machine learning methods that rely on the transfer of human knowledge to an agent have shown to be time efficient for obtaining well performing policies and do not require a reward function. In this context, we analyze the use of human corrective feedback during task execution to learn policies with high-dimensional state spaces, by using the D-COACH framework, and we propose new variants of this framework. D-COACH is a Deep Learning based extension of COACH (COrrective Advice Communicated by Humans), where humans are able to shape policies through corrective advice. The enhanced version of D-COACH, which is proposed in this paper, largely reduces the time and effort of a human for training a policy. Experimental results validate the efficiency of the D-COACH framework in three different problems (simulated and with real robots), and show that its enhanced version reduces the human training effort considerably, and makes it feasible to learn policies within periods of time in which a DRL agent do not reach any improvement.
\end{abstract_eng}


\begin{dedicatoria}
A mi abuelo Ernesto.
\end{dedicatoria}

\begin{thanks}

\end{thanks}

\cleardoublepage
\tableofcontents
\cleardoublepage
\listoftables
\cleardoublepage
\listoffigures

\mainmatter

\begin{intro}
\section{Motivation and Problem Statement}
In  recent years outstanding results in complex decision-making problems have been obtained with Deep Reinforcement Learning (DRL). State of the art algorithms have solved problems with large state spaces and discrete action spaces, such as playing Atari games \cite{atari}, or beating the world champion in GO \cite{Silver2016}, along with low level simulated continuous control tasks in environments such as the ones included in the OpenAI Gym \cite{brockman2016openai} and the DeepMind Control Suite \cite{tassa2018deepmind}. Learning policies, parameterized with Convolutional Neural Networks (CNN) for high-dimensional state spaces, such as raw images, gives agents the possibility to build rich state representations of the environment without feature engineering on the side of the designer (which was always necessary in classical RL). These properties can be really useful in complex robotics problems, giving robots the ability to solve problems using raw visual information. 

Nevertheless, DRL has several limitations when used to address real world systems \cite{Gu2017}. For instance, DRL algorithms require large amounts of data, which means long training times that in contrast to simulated environments cannot be accelerated with more computational power. If somehow this shortcoming was addressed, sometimes the reward function would still pose a problem as it is hard to specify/model and/or to compute in many cases in the real-world. For instance, sometimes additional perception capabilities to the ones of the agent are needed for computing the reward function, since in theory the reward is given ``by the environment``, not be the agent.

In this regard, the transfer of knowledge learned in a simulator to the real-world is a typical solution. However the mismatch between the virtual and real environment, known as ``Reality Gap``, is often problematic \cite{koos2013transferability}. This results in agents that do not perform at their best in the real-world. Thus, it would be preferable to learn/fine-tune policies directly in the real-world.

On the other hand, machine learning methods that rely on the transfer of human knowledge to an agent have shown to be time efficient for obtaining good performance policies. Moreover, some methods do not need expert human teachers for training high performance agents \cite{akrour2011preference,Knox:2009:ISA:1597735.1597738,Celemin2018AnInteractive}. This is why they appear to be good candidates to tackle the DRL real-world issues mentioned before. 

Therefore, in this work, we study the use of human corrective feedback during task execution, to learn policies with high dimensional state spaces, in continuous action problems using CNNs. Our work extends D-COACH \cite{perez2018interactive}, which is a Deep Learning (DL) based extension of the COrrective Advice Communicated by Humans (COACH) framework \cite{Celemin2018AnInteractive}. In the original D-COACH formulation, a demonstration session is required, at the beginning of the training process, for tuning the convolutional layers used for state dimensionality reduction. After that, a fully connected network policy (connected to the previously trained encoder) is interactively trained during task execution with the human corrective feedback, similarly to the human-agent interaction of the original COACH.

In this paper we introduce an enhanced version of D-COACH, which eliminates the need of demonstration sessions and trains the whole CNN simultaneously, reducing the time and effort of the user/coach for teaching a policy.  In D-COACH no reward functions are needed, and the amount of learning episodes are significantly reduced in comparison to alternative DRL approaches. Enhanced D-COACH is validated in three different problems through simulations and real-world scenarios. In each problem, %two % 
the original and enhanced D-COACH are analyzed and compared with the DDPG method. 

ISER

Deep Reinforcement Learning (DRL) has obtained unprecedented results in decision-making problems, such as playing Atari games \cite{Mnih2013}, or beating the world champion in GO \cite{Silver2016}. Nevertheless, in robotic problems, DRL is still limited in applications with real-world systems \cite{Gu2017}. Most of the tasks that have been successfully addressed with DRL have two common characteristics: 1) they have well-specified reward functions, and 2) they require large amounts of trials, which means long training periods (or powerful computers) to obtain a satisfying behavior. These two characteristics can be problematic in cases where 1) the goals of the tasks are poorly defined or hard to specify/model (reward function does not exist), 2) the execution of many trials is not feasible (real systems case) and/or not much computational power or time is available, and 3) sometimes additional external perception is necessary for computing the reward/cost function. 

On the other hand, Machine Learning methods that rely on transfer of human knowledge, Interactive Machine Learning (IML) methods, have shown to be time efficient for obtaining good performance policies and may not require a well-specified reward function; moreover, some methods do not need expert human teachers for training high performance agents \cite{akrour2011preference,Knox:2009:ISA:1597735.1597738,Celemin2018AnInteractive}. In previous years, IML techniques were limited to work with low-dimensional state spaces problems and to the use of function approximation such as linear models of basis functions (choosing a right basis function set was crucial for successful learning), in the same way as RL. But, as DRL have showed, by approximating policies with Deep Neural Networks (DNNs) it is possible to solve problems with high-dimensional state spaces, without the need of feature engineering for preprocessing the states. If the same approach is used in IML, the DRL shortcomings mentioned before can be addressed with the support of human users who participate in the learning process of the agent.

This work proposes to extend the use of human corrective feedback during task execution to learn policies with state spaces of low and high dimensionality in continuous action problems (which is the case for most of the problems in robotics) using deep neural networks.

We combine Deep Learning (DL) with the corrective advice based learning framework called COrrective Advice Communicated by Humans (COACH) \cite{Celemin2018AnInteractive}, thus creating the Deep COACH (D-COACH) framework. In this approach, no reward functions are needed and the amount of learning episodes is significantly reduced in comparison to alternative approaches. D-COACH is validated in three different tasks, two in simulations and one in the real-world.
\section{Objectives}
\subsection{General Objectives}
\subsection{Specific Objectives}
\section{Hypotheses}
\section{Contribution}
\section{Outline}
\end{intro}
\chapter{Theoretical Framework}
\section{Machine Learning}

Machine Learning (ML) is a discipline that studies ways of finding patterns in data by learning from experience. What this mean, is that an algorithm in charge of accomplishing an specific task is able to progressively improve its performance (learn) by comparing its output with data (experience) in a way that indicates how good/bad is doing. If the algorithm is doing good, then the learning process is finished. If is not doing so good, then the algorithm updates its parameters to match (if possible) its output with what is expected.   
\subsection{Deep Learning}

\subsubsection{Autoencoder}

\section{Policy Learning}
\note{POLICY DESCRIPTION}
\subsection{Reinforcement Learning}
\subsubsection{Deep Reinforcement Learning}
\subsection{Learning from Demonstration}
\subsubsection{The Correspondence Problem}
\subsection{Interactive Learning}
\subsubsection{Evaluative Feedback}
\subsubsection{Corrective Feedback}
\section{Function Approximation}
\subsection{Linear Model of Basis Functions}
\subsection{Artificial Neural Networks}
\section{COrrective Advice Communicated by Humans (COACH)}
\chapter{Using Neural Networks to Enhance COACH}
\section{Deep COACH}

\begin{conclusion}
We presented a framework for learning continuous-action policies using corrective feedback to shape DNN policies in high and low dimensional state spaces. This work was inspired in the COrrective Advice Communicated by Humans (COACH) framework, from where two ideas were taken: (1) the use of binary corrective feedback in action spaces for shaping policies, and (2) the use of past corrections to modify the effects of newer ones. In the COACH framework, these ideas were validated in problems with low-dimensional state spaces using linear combination of basis functions (LCBFs) as function approximators. The novelty of this work is that Deep Neural Networks (DNNs) were used instead, combining ideas of the Deep Reinforcement Learning (DRL) era with the ones of COACH, creating Deep COACH (D-COACH). With D-COACH, we showed that by combining Deep Learning with COACH it is possible to extrapolate the ideas of COACH to learn policies in high-dimensional state spaces and in Partially Observable Markov Decision Processes (POMDPs) within the tens of minutes. 

The hypothesis of this thesis was supported with several experiments. We showed that human corrective feedback can be used to learn well performing DNN policies in a time-efficient and reward function free way.

First, D-COACH was validated in a low-dimensional state problem (cart-pole). This is a simple widely-studied problem that is useful to test early versions of sequential decision making learning algorithms. This problem had also been previously solved using COACH, which made it a good candidate to compare the performance of D-COACH with the one of COACH. This experiment showed that D-COACH worked well in low-dimensional state problems and with a performance similar to the one of COACH. 

For high-dimensional state problems, two variations of D-COACH were proposed and compared: online state learning and offline state learning. Both variations obtained similar final performances in the Car Racing and the Duckie Racing problems. The main advantage found in the online state learning version over the offline state version, is that it is an approach that requires less time and effort from the human user. Everything is learned from scratch and interactively, while in the offline state learning case a database of the agent exploring the environment must be obtained and used to train an autoencoder before starting the interactive learning process of the policy. Online state learning D-COACH had an extra validation in a 3DoF arm, where an agent learned to solve the reacher and pusher tasks from scratch.

Finally, a last variation of D-COACH (model-based) for POMDPs in where the observations do not capture time-dependent phenomena from the state was proposed. The approach was to give memory to the agent by adding recurrent layers, LSTMs, to the DNN models. By comparing model-free with model-based D-COACH in low and high dimensional state problems using a simulated teacher, we observed that learning a model of the dynamics of the environment could be crucial in some cases for obtaining well performing policies. Also, that in other cases it may improve the performance of the agent. 

Even though the algorithms proposed in this work were validated and supported the hypothesis of this thesis, we believe that there is still a lot of work and research left to be done in this area. There are some ideas proposed in the COACH framework that we did not study with D-COACH and are worth mentioning:

\begin{itemize}
    \item \textbf{Human Model:} In D-COACH we replaced the Human Model with a replay buffer, as mentioned in Chapter 2. Even though both approaches use information given by past corrections to modify the effects of newer ones, they do not do it in the same way. The Human Model modifies the learning rate of the policy; the replay buffer updates the policy constantly by replaying past corrections. Including a Human Model in D-COACH could help with the dilemma of setting either a too large or too small magnitude of the learning rate when updating the policy with SGD. In this case, we did not include a Human Model because this would have meant to add a second DNN model in charge of learning it. Thus, the overhead of D-COACH would have increased and for a first approach we prioritized a lighter model.
    \item \textbf{Credit assigner:} Module proposed in TAMER approaches \cite{Knox:2009:ISA:1597735.1597738} which COACH adopted. This module associates feedback not only with the last state-action pair, but with past state-action pairs as well. The objective is to characterize the the human delay with a probability that weights correction signals with a sequence of state-action pairs. This could help with the data-efficiency of D-COACH. Nevertheless, given that D-COACH was able to work fine without this module, studying the advantages of adding it to the framework was left for future work. 
    \item \textbf{D-COACH + DRL:} In \cite{celemin2018fast}, a hybrid RL framework with COACH is proposed. The basic idea is to use corrective feedback along with RL algorithms in order to speed up the learning process. The same concept could be applied and studied with D-COACH. 
\end{itemize}

In contrast, the are some shortcomings that D-COACH presents that would be beneficial to study in future research:
\begin{itemize}
    \item \textbf{High-dimensional action spaces:} One of the main shortcomings of D-COACH was inherited from COACH. Both approaches are limited to work in problems with low-dimensional action spaces due to that humans must provide corrections in the action space. If the action space of a problem is too large, then is not intuitive for a human to give feedback, and he/she may not be able to correct the agent. Modules that interpret feedback from a correction space to the agent's action space can be added to tackle this shortcoming, such using inverse kinematics modules. But this is not always possible or trivial to do, so more research in this area could  enhance the capabilities of D-COACH.
    \item \textbf{Experience Replay size:} Given that the size of the replay buffer of D-COACH is limited due to its on-policy nature, it may be challenging (or not possible) to solve problems that require more complex decision-making strategies that the ones tested in this thesis. This is because in more complex scenarios, agents may need to store corrections in memory for a longer time than what the buffer is able to do, due to its limited size. Thus, valuable corrections would be forgotten, affecting the performance of the agent. Developing strategies to overcame this limitation could be key for using D-COACH in more complex settings. 
\end{itemize}

Beyond the limitations of D-COACH and the future research that can be done in this area, the variations of D-COACH presented in this work showed to be a valid alternative for teaching robots to solve sequential decision-making problems using DNN policies. Human teachers were able to interact with real-world platforms (Duckie Racing and 3DoF arm) and guide them through the learning process for solving tasks.
\end{conclusion}


\nocite{*}
\bibliographystyle{plain}
\bibliography{bibliografia}
\end{document}
